\documentclass[a4paper,12pt]{article} % тип документа

% Поля страниц
\usepackage[left=2.5cm,right=2.5cm,
    top=2cm,bottom=2cm,bindingoffset=0cm]{geometry}
 
    
%Отступ после заголовка    
\usepackage{indentfirst}


% Рисунки
\usepackage{floatrow,graphicx,calc}
\usepackage{wrapfig}

% Создаёем новый разделитель
\DeclareFloatSeparators{mysep}{\hspace{1cm}}

% Ссылки?
\usepackage[unicode, pdftex]{hyperref} % подключаем hyperref
\usepackage{hyperref}
\usepackage[rgb]{xcolor}
\hypersetup{				% Гиперссылки
    colorlinks=true,       	% false: ссылки в рамках
	urlcolor=blue          % на URL
}


%  Русский язык
\usepackage[T2A]{fontenc}			% кодировка
\usepackage[utf8]{inputenc}			% кодировка исходного текста
\usepackage[english,russian]{babel}	% локализация и переносы


% Математика
\usepackage{amsmath,amsfonts,amssymb,amsthm,mathtools}


% Что-то 
\usepackage{wasysym}

\begin {document}

\begin{titlepage}
\newcommand{\HRule}{\rule{\linewidth}{0.3 mm}} % Defines a Hnew command for the horizontal lines, change thickness here

\center % Center everything on the page
 
%----------------------------------------------------------------------------------------
%	HEADING SECTIONS
%----------------------------------------------------------------------------------------

\textsc{\Large Московский физико-технический институт }\\[1.5cm] % Name of your university/college
\textsc{\Large Факультет аэрокосмических технологий}\\[0.5cm] % Major heading such as course name
\textsc{\large Лабораторная работа }\\[0.5cm] % Minor heading such as course title

%----------------------------------------------------------------------------------------
%	TITLE SECTION
%----------------------------------------------------------------------------------------

\HRule \\[0.4cm]
{ \huge \bfseries Устойчивость стержней }\\[0.4cm] % Title of your document
\HRule \\[1.5cm]
 
%----------------------------------------------------------------------------------------
%	AUTHOR SECTION
%----------------------------------------------------------------------------------------

\begin{minipage}{0.4\textwidth}
\begin{flushleft} \large
\emph{Автор:}\\ Артем \textsc{Овчинников} % Your name
\end{flushleft}
\end{minipage}
\begin{minipage}{0.4\textwidth}
\begin{flushright} \large
\emph{Преподаватель:} \\
??? \textsc{???} % Supervisor's Name
\end{flushright}
\end{minipage}\\[4cm]
%	DATE SECTION
%----------------------------------------------------------------------------------------

{\large \today}\\[2cm] % Date, change the \today to a set date if you want to be precise

%----------------------------------------------------------------------------------------
%	LOGO SECTION
%----------------------------------------------------------------------------------------

 
%----------------------------------------------------------------------------------------

\vfill % Fill the rest of the page with whitespace

\end{titlepage}
\tableofcontents

%----------------------------------------------------------------------------------------

\newpage
\section{Аннотация}
В данной работе изучена теория Эйлера устойчивости сжатого стержня в теории и на опыте, приведено сравнение результатов.


\section{Теоретические сведения}
Условие равновесия части стержня:
\begin{equation}
    M(x) = -Py
\end{equation}
где $M(x)$ - изгибающий момент в сечении, $P$ - сжимающая сила.

Из теории изгиба балки (при малом $y$):
\begin{equation}
    y" \approx \frac{M(x)}{EJ}
\end{equation}

Решая полученное дифференциальное уравнение ($y(0)=y(l)=0$):
\begin{equation}
    P_{cr} = \frac{\pi^2 n^2  EJ}{l^2}
\end{equation}


\section{Методика измерений}
Под руководством преподавателя расположите и закрепите
кронштейны с роликами на раме. Нагрузите стержни, используя винты, до
потери устойчивости. Для этого снимите показания датчиков пере-
мещения для 8—10 значений силы и отложите полученные значения
на графике. С помощью графика определите критическую силу.


\section{Используемое оборудование}
На лабораторном стенде смонтированы два длинных стержня прямоугольного сечения (линейки). На стержни нанесена миллиметровая шкала. Один стержень закреплен консольно, другой имеет на конце ролик, который может свободно проскальзывать по опоре. \\
Передача сжимающей нагрузки производится винтами, величину сжимающей силы регистрируют датчики силы. Контроль перемещения точки приложения силы осуществляется с помощью датчиков перемещения.



\section{Результаты измерений и обработка данных}
\section{Обсуждение результатов}
\section{Заключение}

\end{document}
